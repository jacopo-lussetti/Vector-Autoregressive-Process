% Options for packages loaded elsewhere
\PassOptionsToPackage{unicode}{hyperref}
\PassOptionsToPackage{hyphens}{url}
\documentclass[
]{article}
\usepackage{xcolor}
\usepackage[margin=1in]{geometry}
\usepackage{amsmath,amssymb}
\setcounter{secnumdepth}{-\maxdimen} % remove section numbering
\usepackage{iftex}
\ifPDFTeX
  \usepackage[T1]{fontenc}
  \usepackage[utf8]{inputenc}
  \usepackage{textcomp} % provide euro and other symbols
\else % if luatex or xetex
  \usepackage{unicode-math} % this also loads fontspec
  \defaultfontfeatures{Scale=MatchLowercase}
  \defaultfontfeatures[\rmfamily]{Ligatures=TeX,Scale=1}
\fi
\usepackage{lmodern}
\ifPDFTeX\else
  % xetex/luatex font selection
\fi
% Use upquote if available, for straight quotes in verbatim environments
\IfFileExists{upquote.sty}{\usepackage{upquote}}{}
\IfFileExists{microtype.sty}{% use microtype if available
  \usepackage[]{microtype}
  \UseMicrotypeSet[protrusion]{basicmath} % disable protrusion for tt fonts
}{}
\makeatletter
\@ifundefined{KOMAClassName}{% if non-KOMA class
  \IfFileExists{parskip.sty}{%
    \usepackage{parskip}
  }{% else
    \setlength{\parindent}{0pt}
    \setlength{\parskip}{6pt plus 2pt minus 1pt}}
}{% if KOMA class
  \KOMAoptions{parskip=half}}
\makeatother
\usepackage{color}
\usepackage{fancyvrb}
\newcommand{\VerbBar}{|}
\newcommand{\VERB}{\Verb[commandchars=\\\{\}]}
\DefineVerbatimEnvironment{Highlighting}{Verbatim}{commandchars=\\\{\}}
% Add ',fontsize=\small' for more characters per line
\usepackage{framed}
\definecolor{shadecolor}{RGB}{248,248,248}
\newenvironment{Shaded}{\begin{snugshade}}{\end{snugshade}}
\newcommand{\AlertTok}[1]{\textcolor[rgb]{0.94,0.16,0.16}{#1}}
\newcommand{\AnnotationTok}[1]{\textcolor[rgb]{0.56,0.35,0.01}{\textbf{\textit{#1}}}}
\newcommand{\AttributeTok}[1]{\textcolor[rgb]{0.13,0.29,0.53}{#1}}
\newcommand{\BaseNTok}[1]{\textcolor[rgb]{0.00,0.00,0.81}{#1}}
\newcommand{\BuiltInTok}[1]{#1}
\newcommand{\CharTok}[1]{\textcolor[rgb]{0.31,0.60,0.02}{#1}}
\newcommand{\CommentTok}[1]{\textcolor[rgb]{0.56,0.35,0.01}{\textit{#1}}}
\newcommand{\CommentVarTok}[1]{\textcolor[rgb]{0.56,0.35,0.01}{\textbf{\textit{#1}}}}
\newcommand{\ConstantTok}[1]{\textcolor[rgb]{0.56,0.35,0.01}{#1}}
\newcommand{\ControlFlowTok}[1]{\textcolor[rgb]{0.13,0.29,0.53}{\textbf{#1}}}
\newcommand{\DataTypeTok}[1]{\textcolor[rgb]{0.13,0.29,0.53}{#1}}
\newcommand{\DecValTok}[1]{\textcolor[rgb]{0.00,0.00,0.81}{#1}}
\newcommand{\DocumentationTok}[1]{\textcolor[rgb]{0.56,0.35,0.01}{\textbf{\textit{#1}}}}
\newcommand{\ErrorTok}[1]{\textcolor[rgb]{0.64,0.00,0.00}{\textbf{#1}}}
\newcommand{\ExtensionTok}[1]{#1}
\newcommand{\FloatTok}[1]{\textcolor[rgb]{0.00,0.00,0.81}{#1}}
\newcommand{\FunctionTok}[1]{\textcolor[rgb]{0.13,0.29,0.53}{\textbf{#1}}}
\newcommand{\ImportTok}[1]{#1}
\newcommand{\InformationTok}[1]{\textcolor[rgb]{0.56,0.35,0.01}{\textbf{\textit{#1}}}}
\newcommand{\KeywordTok}[1]{\textcolor[rgb]{0.13,0.29,0.53}{\textbf{#1}}}
\newcommand{\NormalTok}[1]{#1}
\newcommand{\OperatorTok}[1]{\textcolor[rgb]{0.81,0.36,0.00}{\textbf{#1}}}
\newcommand{\OtherTok}[1]{\textcolor[rgb]{0.56,0.35,0.01}{#1}}
\newcommand{\PreprocessorTok}[1]{\textcolor[rgb]{0.56,0.35,0.01}{\textit{#1}}}
\newcommand{\RegionMarkerTok}[1]{#1}
\newcommand{\SpecialCharTok}[1]{\textcolor[rgb]{0.81,0.36,0.00}{\textbf{#1}}}
\newcommand{\SpecialStringTok}[1]{\textcolor[rgb]{0.31,0.60,0.02}{#1}}
\newcommand{\StringTok}[1]{\textcolor[rgb]{0.31,0.60,0.02}{#1}}
\newcommand{\VariableTok}[1]{\textcolor[rgb]{0.00,0.00,0.00}{#1}}
\newcommand{\VerbatimStringTok}[1]{\textcolor[rgb]{0.31,0.60,0.02}{#1}}
\newcommand{\WarningTok}[1]{\textcolor[rgb]{0.56,0.35,0.01}{\textbf{\textit{#1}}}}
\usepackage{graphicx}
\makeatletter
\newsavebox\pandoc@box
\newcommand*\pandocbounded[1]{% scales image to fit in text height/width
  \sbox\pandoc@box{#1}%
  \Gscale@div\@tempa{\textheight}{\dimexpr\ht\pandoc@box+\dp\pandoc@box\relax}%
  \Gscale@div\@tempb{\linewidth}{\wd\pandoc@box}%
  \ifdim\@tempb\p@<\@tempa\p@\let\@tempa\@tempb\fi% select the smaller of both
  \ifdim\@tempa\p@<\p@\scalebox{\@tempa}{\usebox\pandoc@box}%
  \else\usebox{\pandoc@box}%
  \fi%
}
% Set default figure placement to htbp
\def\fps@figure{htbp}
\makeatother
\setlength{\emergencystretch}{3em} % prevent overfull lines
\providecommand{\tightlist}{%
  \setlength{\itemsep}{0pt}\setlength{\parskip}{0pt}}
\usepackage{bookmark}
\IfFileExists{xurl.sty}{\usepackage{xurl}}{} % add URL line breaks if available
\urlstyle{same}
\hypersetup{
  pdftitle={Regularised Cov. Estimation},
  pdfauthor={Jacopo Lussetti},
  hidelinks,
  pdfcreator={LaTeX via pandoc}}

\title{Regularised Cov. Estimation}
\author{Jacopo Lussetti}
\date{2025-10-27}

\begin{document}
\maketitle

This report is aiming at replicating the methodology by
\cite{basu2015regularized} on estimating covaiane matrices and their
precision matrices for high-dimensional data.

\section{Exercise 1}\label{exercise-1}

\#Data Generation To simulate the stochastic regression model, we will
generate data from our code that allows to randomly generate an uppper
triangle coefficient matrix. Then we will simulate the data with a
VAR(1) model. Then we will proceed to use a Gaussian VAR(1) model
\textbackslash{}
\(X^t=AX^{t-1} + \epsilon^t, \quad \epsilon^t \overset{\text{i.i.d.}}{\sim}\mathcal{N}(\alpha I_{Kp})\),
where \(\alpha =0.2\)

\begin{Shaded}
\begin{Highlighting}[]
\CommentTok{\#we define first of all the various functions}
\DocumentationTok{\#\#Simulation upper triangular matrix}
\DocumentationTok{\#\#\# To ensure that process is stable, we need to check the abs of eigenvalue}
\NormalTok{stab\_test }\OtherTok{\textless{}{-}} \ControlFlowTok{function}\NormalTok{(kp, A, }\AttributeTok{tol =} \FloatTok{1e{-}8}\NormalTok{)}
\NormalTok{\{}
  \ControlFlowTok{if}\NormalTok{ (}\SpecialCharTok{!}\FunctionTok{is.matrix}\NormalTok{(A) }\SpecialCharTok{||} \FunctionTok{nrow}\NormalTok{(A) }\SpecialCharTok{!=} \FunctionTok{ncol}\NormalTok{(A)) \{}
    \FunctionTok{stop}\NormalTok{(}\StringTok{"The matrix is not square"}\NormalTok{)}
\NormalTok{  \}}
\NormalTok{  eig }\OtherTok{\textless{}{-}} \FunctionTok{eigen}\NormalTok{(A, }\AttributeTok{only.values =} \ConstantTok{TRUE}\NormalTok{)}\SpecialCharTok{$}\NormalTok{values  }\CommentTok{\# computing the eigenvalues}
  
  \ControlFlowTok{for}\NormalTok{ (i }\ControlFlowTok{in} \DecValTok{1}\SpecialCharTok{:}\FunctionTok{length}\NormalTok{(eig)) \{     }
    \ControlFlowTok{if}\NormalTok{ (}\FunctionTok{Mod}\NormalTok{(eig[i]) }\SpecialCharTok{\textgreater{}=} \DecValTok{1} \SpecialCharTok{{-}}\NormalTok{ tol) \{ }
      \FunctionTok{return}\NormalTok{(}\ConstantTok{FALSE}\NormalTok{)               }
\NormalTok{    \}}
\NormalTok{  \}}
  \FunctionTok{return}\NormalTok{(}\ConstantTok{TRUE}\NormalTok{)}
\NormalTok{\}}

\CommentTok{\#function to compute companion matrix}
\NormalTok{comp\_mtrx }\OtherTok{\textless{}{-}} \ControlFlowTok{function}\NormalTok{(AA)\{}
    \DocumentationTok{\#\# AA is a K x Kp matrix, so we are able to derive p in the following way}
\NormalTok{    K }\OtherTok{\textless{}{-}} \FunctionTok{nrow}\NormalTok{(AA)}
\NormalTok{    Kp }\OtherTok{\textless{}{-}} \FunctionTok{ncol}\NormalTok{(AA)      }
\NormalTok{    p }\OtherTok{\textless{}{-}}\NormalTok{ Kp}\SpecialCharTok{/}\NormalTok{K}
    
    \CommentTok{\# Create the empty companion matrix Kp x Kp}
\NormalTok{    C }\OtherTok{\textless{}{-}} \FunctionTok{matrix}\NormalTok{(}\DecValTok{0}\NormalTok{, }\AttributeTok{nrow=}\NormalTok{Kp, }\AttributeTok{ncol=}\NormalTok{Kp)}

\NormalTok{    C[}\DecValTok{1}\SpecialCharTok{:}\NormalTok{K,] }\OtherTok{\textless{}{-}}\NormalTok{ AA}
    
    \CommentTok{\# Add ones on the K{-}th sub{-}diagonal}
    \ControlFlowTok{if}\NormalTok{ (p}\SpecialCharTok{\textgreater{}}\DecValTok{1}\NormalTok{)}
\NormalTok{        C[(K}\SpecialCharTok{+}\DecValTok{1}\NormalTok{)}\SpecialCharTok{:}\NormalTok{Kp, }\DecValTok{1}\SpecialCharTok{:}\NormalTok{(Kp}\SpecialCharTok{{-}}\NormalTok{K)] }\OtherTok{\textless{}{-}} \FunctionTok{diag}\NormalTok{(Kp }\SpecialCharTok{{-}}\NormalTok{ K)}
    \FunctionTok{return}\NormalTok{(C)}
\NormalTok{\}}
\CommentTok{\#}

\DocumentationTok{\#\# VAR(p) process Simulator}
\NormalTok{var\_sim }\OtherTok{\textless{}{-}} \ControlFlowTok{function}\NormalTok{(AA, nu, Sigma\_u, nSteps, y0) \{}
\NormalTok{    K }\OtherTok{\textless{}{-}} \FunctionTok{nrow}\NormalTok{(Sigma\_u)}
\NormalTok{    Kp }\OtherTok{\textless{}{-}} \FunctionTok{ncol}\NormalTok{(AA)}
\NormalTok{    p }\OtherTok{\textless{}{-}}\NormalTok{ Kp}\SpecialCharTok{/}\NormalTok{K}
        
    \ControlFlowTok{if}\NormalTok{ (p }\SpecialCharTok{\textgreater{}} \DecValTok{1}\NormalTok{) \{}
\NormalTok{        C }\OtherTok{\textless{}{-}} \FunctionTok{comp\_mtrx}\NormalTok{(AA) }\CommentTok{\# form the  companion matrix of the var(p) process}
\NormalTok{    \} }\ControlFlowTok{else}\NormalTok{ \{}
\NormalTok{        C }\OtherTok{\textless{}{-}}\NormalTok{ AA  }
\NormalTok{    \}}
\NormalTok{    y\_t }\OtherTok{\textless{}{-}} \FunctionTok{matrix}\NormalTok{(}\DecValTok{0}\NormalTok{, }\AttributeTok{nrow =}\NormalTok{  nSteps, }\AttributeTok{ncol=}\NormalTok{Kp) }\CommentTok{\#trajectories matrix nSteps x Kp}
\NormalTok{    y\_t[}\DecValTok{1}\NormalTok{, }\DecValTok{1}\SpecialCharTok{:}\NormalTok{Kp] }\OtherTok{\textless{}{-}}\NormalTok{ y0 }\CommentTok{\#add initial value to initiate the simulation}
\NormalTok{    noise }\OtherTok{\textless{}{-}} \FunctionTok{mvrnorm}\NormalTok{(}\AttributeTok{n =}\NormalTok{ nSteps, }\AttributeTok{mu =} \FunctionTok{rep}\NormalTok{(}\DecValTok{0}\NormalTok{, K), }\AttributeTok{Sigma =}\NormalTok{ Sigma\_u) }\CommentTok{\#assuming that }
    \CommentTok{\#residuals follow a multivariate normal distribution    }
    
    \ControlFlowTok{for}\NormalTok{ (t }\ControlFlowTok{in} \DecValTok{2}\SpecialCharTok{:}\NormalTok{nSteps) \{}
\NormalTok{        y\_t[t, ] }\OtherTok{\textless{}{-}}\NormalTok{ C }\SpecialCharTok{\%*\%}\NormalTok{ y\_t[t}\DecValTok{{-}1}\NormalTok{, ]}
\NormalTok{        y\_t[t, }\DecValTok{1}\SpecialCharTok{:}\NormalTok{K] }\OtherTok{\textless{}{-}}\NormalTok{ y\_t[t, }\DecValTok{1}\SpecialCharTok{:}\NormalTok{K] }\SpecialCharTok{+}\NormalTok{ nu }\SpecialCharTok{+}\NormalTok{ noise[t,]}
\NormalTok{    \}}
    
\NormalTok{    y\_t }\OtherTok{\textless{}{-}} \FunctionTok{zoo}\NormalTok{(y\_t[,}\DecValTok{1}\SpecialCharTok{:}\NormalTok{K], }\DecValTok{1}\SpecialCharTok{:}\NormalTok{nSteps)  }
    \FunctionTok{return}\NormalTok{(y\_t)}
\NormalTok{  \}}
\end{Highlighting}
\end{Shaded}

We will now replicate figure 1 from \cite{basu2015regularized}. He
provided the following values for the spectral radius \(\rho(A)\) and
fixed diagonal elements of the generating coefficient matrix \(A\) to
generate the VAR(1) process.

\begin{table}[h!]
\centering
\begin{tabular}{c|c}
$\alpha$ & $\rho(A)$ \\
\hline
0.2 & 0.2 \\
0.2 & 0.92 \\
0.2 & 0.96 \\
0.2 & 1 \\
0.2 & 1.01 \\
0.2 & 1.02 \\
0.2 & 1.03 \\
\end{tabular}
\end{table}

To ensure that \(\rho(A)\) is fixed to the desired value, we will scale
the matrix \(A\) by the following scalar:

\begin{align*}
\|A\|_2 &= \sqrt{\rho(A^T A)} = a \\
\alpha &= \sqrt{\frac{a}{\rho(A^T A)}} = \frac{\sqrt{a}}{\sqrt{\rho(A^T A)}}
\end{align*}

\begin{Shaded}
\begin{Highlighting}[]
\CommentTok{\#we first simulate the coef for the predictors}
\NormalTok{K}\OtherTok{\textless{}{-}}\DecValTok{200}
\NormalTok{A\_coef}\OtherTok{\textless{}{-}}\FunctionTok{matrix}\NormalTok{(}\DecValTok{0}\DataTypeTok{L}\NormalTok{,}\AttributeTok{nrow=}\NormalTok{K, }\AttributeTok{ncol=}\NormalTok{K)}
\CommentTok{\#we select the upper triangle element}
\NormalTok{d}\OtherTok{\textless{}{-}}\FunctionTok{dim}\NormalTok{(A\_coef)}
\NormalTok{upper\_indx}\OtherTok{\textless{}{-}} \FunctionTok{which}\NormalTok{(}\FunctionTok{row}\NormalTok{(A\_coef)}\SpecialCharTok{\textless{}}\FunctionTok{col}\NormalTok{(A\_coef)) }\CommentTok{\#index upper triangle element}
\CommentTok{\#we need to ensure that the spectral norm is fixed to a certain value. }
\CommentTok{\#we consider the the first case when 0.2 from figure 1}
\NormalTok{t}\OtherTok{\textless{}{-}}\DecValTok{0}
\ControlFlowTok{repeat}\NormalTok{\{}
\NormalTok{  A\_coef[upper\_indx]}\OtherTok{\textless{}{-}}\FunctionTok{rnorm}\NormalTok{(}\FunctionTok{length}\NormalTok{(upper\_indx), }\AttributeTok{mean=}\DecValTok{0}\NormalTok{, }\AttributeTok{sd=}\FloatTok{0.2}\NormalTok{)}
  \CommentTok{\#we add diagonal values=0.2 as requested from the paper}
  \FunctionTok{diag}\NormalTok{(A\_coef)}\OtherTok{\textless{}{-}}\FloatTok{0.2}
  \CommentTok{\#check spectral norm}
\NormalTok{  spec\_norm}\OtherTok{\textless{}{-}}\FunctionTok{sqrt}\NormalTok{(}\FunctionTok{max}\NormalTok{(}\FunctionTok{eigen}\NormalTok{(}\FunctionTok{t}\NormalTok{(A\_coef)}\SpecialCharTok{\%*\%}\NormalTok{A\_coef)}\SpecialCharTok{$}\NormalTok{values))}
  \ControlFlowTok{if}\NormalTok{ (spec\_norm}\FloatTok{{-}0.2} \SpecialCharTok{\textless{}}\FloatTok{1e{-}2}\NormalTok{)\{}
    \ControlFlowTok{break}
\NormalTok{  \}}
  
\NormalTok{  t}\OtherTok{\textless{}{-}}\NormalTok{t}\SpecialCharTok{+}\DecValTok{1}
  \ControlFlowTok{if}\NormalTok{(t}\SpecialCharTok{\textgreater{}}\DecValTok{100}\NormalTok{)\{}
    \ControlFlowTok{break}
\NormalTok{  \}}
\NormalTok{\}}
\CommentTok{\#we check also stability}
\NormalTok{prod}\OtherTok{\textless{}{-}}\FunctionTok{t}\NormalTok{(A\_coef)}\SpecialCharTok{\%*\%}\NormalTok{A\_coef}
\NormalTok{rho}\OtherTok{\textless{}{-}}\FunctionTok{max}\NormalTok{(}\FunctionTok{abs}\NormalTok{(}\FunctionTok{eigen}\NormalTok{(prod)}\SpecialCharTok{$}\NormalTok{values))}
\FunctionTok{print}\NormalTok{(}\FunctionTok{sqrt}\NormalTok{(rho))}
\end{Highlighting}
\end{Shaded}

\begin{verbatim}
## [1] 4.528965
\end{verbatim}

\begin{Shaded}
\begin{Highlighting}[]
\CommentTok{\#now we co}
\NormalTok{nu}\OtherTok{\textless{}{-}}\FunctionTok{rep}\NormalTok{(}\DecValTok{0}\NormalTok{,K)}
\NormalTok{cov\_res}\OtherTok{\textless{}{-}}\FunctionTok{diag}\NormalTok{(}\FloatTok{0.5}\NormalTok{, }\AttributeTok{nrow=}\NormalTok{K, }\AttributeTok{ncol=}\NormalTok{K)}
\NormalTok{y\_0}\OtherTok{\textless{}{-}}\FunctionTok{rep}\NormalTok{(}\DecValTok{0}\NormalTok{,K)}
\CommentTok{\#trial  for small n, to then complete the simulation}
\NormalTok{predictor}\OtherTok{\textless{}{-}}\FunctionTok{var\_sim}\NormalTok{(}\AttributeTok{AA=}\NormalTok{A\_coef, }\AttributeTok{nu=}\NormalTok{nu, }\AttributeTok{Sigma\_u=}\NormalTok{cov\_res, }\AttributeTok{nSteps=}\DecValTok{50}\NormalTok{, }\AttributeTok{y0=}\NormalTok{y\_0)}
\CommentTok{\#now we generate the sparse true coef matrix }
\CommentTok{\# k\textless{}{-}15 \# 15 numbers different from zero}
\CommentTok{\# set.seed(1234)}
\CommentTok{\# beta\_star\textless{}{-} matrix(0L, nrow=k, ncol=k}
\end{Highlighting}
\end{Shaded}

\printbibliography

\end{document}
